\documentclass[lettersize,journal]{IEEEtran}
\usepackage{amsmath,amsfonts}
\usepackage{algorithmic}
\usepackage{array}
\usepackage[caption=false,font=normalsize,labelfont=sf,textfont=sf]{subfig}
\usepackage{textcomp}
\usepackage{stfloats}
\usepackage{url}
\usepackage{verbatim}
\usepackage{graphicx}
\hyphenation{op-tical net-works semi-conduc-tor IEEE-Xplore}
\def\BibTeX{{\rm B\kern-.05em{\sc i\kern-.025em b}\kern-.08em
    T\kern-.1667em\lower.7ex\hbox{E}\kern-.125emX}}
\usepackage{balance}
\begin{document}
\title{Phân Tích Dân Số Việt Nam Giai Đoạn 1950-1985: Sự Bùng Nổ và Thay Đổi Cấu Trúc Tuổi}
\author{Báo Cáo Phân Tích Dân Số
\thanks{Báo cáo này phân tích dữ liệu dân số Việt Nam giai đoạn 1950-1985, tập trung vào sự bùng nổ dân số và thay đổi cấu trúc các nhóm tuổi.}}

\markboth{Journal of Population Studies,~Vol.~1, No.~1, 2025}%
{Phân Tích Dân Số Việt Nam Giai Đoạn 1950-1985}

\maketitle

\begin{abstract}
Giai đoạn 1950-1985 là thời kỳ chứng kiến sự bùng nổ về quy mô dân số và những thay đổi đáng kể trong cấu trúc các nhóm tuổi tại Việt Nam. Báo cáo này phân tích chi tiết sự gia tăng dân số từ 25,1 triệu người (1950) lên hơn 59 triệu người (1985), với sự tăng trưởng vượt trội của nhóm trẻ em (0-14 tuổi) từ 32,8\% lên 40\% tổng dân số. Phân tích cho thấy Việt Nam có cấu trúc dân số rất trẻ với hình tháp dân số đáy rộng, điển hình của quốc gia đang phát triển có mức sinh cao.
\end{abstract}

\begin{IEEEkeywords}
Dân số Việt Nam, cấu trúc tuổi, bùng nổ dân số, nhóm tuổi 0-14, nhóm tuổi lao động, nhóm người cao tuổi, 1950-1985
\end{IEEEkeywords}

\section{Giới thiệu}
\IEEEPARstart{G}{iai} đoạn 1950-1985 đánh dấu một trong những thời kỳ quan trọng nhất trong lịch sử dân số Việt Nam, với sự thay đổi sâu sắc về quy mô và cấu trúc dân số. Trong vòng 35 năm, tổng dân số Việt Nam đã tăng trưởng mạnh mẽ, từ khoảng 25,1 triệu người (1950) lên đến hơn 59 triệu người (1985), cho thấy quy mô dân số đã tăng hơn gấp đôi chỉ trong hơn ba thập kỷ.

Báo cáo này cung cấp phân tích chi tiết về sự thay đổi trong cấu trúc các nhóm tuổi chính, tập trung vào ba nhóm: trẻ em (0-14 tuổi), lao động (15-64 tuổi), và người cao tuổi (65+ tuổi). Phân tích này giúp hiểu rõ đặc điểm dân số Việt Nam trong giai đoạn này và cơ sở cho các chính sách phát triển kinh tế-xã hội.

\section{Tổng Quan Sự Gia Tăng Dân Số}
\noindent Sự gia tăng dân số Việt Nam giai đoạn 1950-1985 thể hiện qua các số liệu sau:

\begin{itemize}
\item Năm 1950: Khoảng 25,1 triệu người
\item Năm 1985: Hơn 59 triệu người
\item Tỷ lệ tăng trưởng: Hơn gấp đôi trong 35 năm
\item Tốc độ tăng trung bình: Gần 1 triệu người/năm
\end{itemize}

Sự tăng trưởng này phản ánh mức sinh cao và giảm tỷ lệ tử vong nhờ cải thiện y tế và điều kiện sống, tạo nên một giai đoạn "trẻ hóa" dân số đặc trưng.

\section{Phân Tích Theo Các Nhóm Tuổi Chính}

\subsection{Nhóm tuổi trẻ em (0-14 tuổi): Sự bùng nổ mạnh mẽ}
\noindent Đây là nhóm có sự gia tăng ấn tượng nhất về cả số lượng tuyệt đối và tỷ trọng:

\begin{itemize}
\item Năm 1950: Khoảng 8,2 triệu người, chiếm khoảng 32,8\% tổng dân số
\item Năm 1980: Số lượng tăng vọt lên 21,1 triệu người
\item Năm 1985: Đạt mức 23,6 triệu người, chiếm tỷ trọng khoảng 40\% tổng dân số
\end{itemize}

Việc tỷ trọng nhóm trẻ em tăng từ 32,8\% lên 40\% cho thấy Việt Nam trong giai đoạn này có mức sinh rất cao, tạo nên một cấu trúc dân số rất trẻ.

\subsection{Nhóm tuổi lao động (15-64 tuổi): Tăng trưởng ổn định nhưng tỷ trọng giảm}
\noindent Nhóm lao động có sự phát triển như sau:

\begin{itemize}
\item Năm 1950: Khoảng 15,8 triệu người, chiếm tỷ trọng cao nhất với khoảng 63,1\%
\item Năm 1985: Số lượng tăng lên 32,2 triệu người, nhưng tỷ trọng giảm xuống còn khoảng 54,6\%
\end{itemize}

Mặc dù lực lượng lao động tăng gấp đôi về số lượng, nhưng gánh nặng phụ thuộc (tỷ lệ người trẻ và người già so với người lao động) đã tăng lên do số lượng trẻ em tăng quá nhanh.

\subsection{Nhóm người cao tuổi (65+ tuổi): Duy trì tỷ trọng thấp}
\noindent Nhóm người cao tuổi có sự thay đổi:

\begin{itemize}
\item Năm 1950: Khoảng 1,03 triệu người, chiếm khoảng 4,1\% dân số
\item Năm 1985: Tăng lên 3,14 triệu người, chiếm khoảng 5,3\% dân số
\end{itemize}

Mặc dù số lượng người già tăng gấp 3 lần, nhưng tỷ trọng của họ trong tổng dân số vẫn duy trì ở mức rất thấp (dưới 6\%). Điều này khẳng định Việt Nam trong giai đoạn 1950-1985 hoàn toàn chưa bước vào quá trình già hóa dân số.

\section{Đặc Điểm Cấu Trúc Dân Số Giai Đoạn Này}

\subsection{Cấu trúc dân số trẻ}
\noindent Với tỷ trọng nhóm 0-14 tuổi lên đến 40\% vào năm 1985, Việt Nam có hình tháp dân số đáy rộng, điển hình của các quốc gia đang phát triển có mức sinh cao. Cấu trúc này tạo ra:

\begin{itemize}
\item Lực lượng lao động dồi dào trong tương lai
\item Gánh nặng phụ thuộc cao hiện tại
\item Nhu cầu lớn về giáo dục và y tế cho trẻ em
\end{itemize}

\subsection{Tốc độ gia tăng nhanh}
\noindent Trung bình mỗi năm trong giai đoạn này, dân số tăng thêm gần 1 triệu người. Riêng nhóm trẻ em (0-14) có tốc độ tăng trưởng nhanh nhất, làm trẻ hóa cơ cấu dân số một cách mạnh mẽ.

\section{Kết Luận}
\noindent Giai đoạn 1950-1985 là thời kỳ "trẻ hóa" dân số tại Việt Nam, với sự bùng nổ đặc biệt của nhóm trẻ em (0-14 tuổi). Các đặc điểm chính bao gồm:

\begin{itemize}
\item Dân số tăng hơn gấp đôi trong 35 năm
\item Tỷ trọng trẻ em tăng từ 32,8\% lên 40\%
\item Cấu trúc dân số rất trẻ với tháp dân số đáy rộng
\item Chưa có dấu hiệu già hóa dân số
\end{itemize}

Đây là giai đoạn hoàn toàn trái ngược với xu hướng già hóa nhanh chóng mà chúng ta đang thấy trong những thập kỷ gần đây và dự báo cho tương lai. Hiểu rõ đặc điểm dân số giai đoạn này có ý nghĩa quan trọng cho việc phân tích lịch sử phát triển và hoạch định chính sách hiện tại.

\begin{thebibliography}{1}

\bibitem{pop1}
General Statistics Office of Vietnam, \textit{Population and Housing Census Reports}, 1950-1985.

\bibitem{pop2}
United Nations Population Division, \textit{World Population Prospects}, Vietnam Demographic Data, 1950-1985.

\bibitem{pop3}
Institute of Sociology, Vietnam Academy of Social Sciences, \textit{Vietnam Population Dynamics in the 20th Century}, Hanoi, 1990.

\end{thebibliography}

\end{document}
